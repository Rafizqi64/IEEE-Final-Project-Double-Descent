\subsection{Dataset}
The Patch-CAMELYON dataset was employed in this study, which is a publicly available dataset containing over 327,000 histopathology images. These images are used for the automated detection and classification of metastatic breast cancer in lymph node tissues. Each 96x96 pixel color image represents a tissue sample from a histopathology slide of a patient, labeled as either positive or negative for the presence of metastatic cancer.

\subsection{Model Development}
A custom Convolutional Neural Network (CNN)-based approach was developed for the analysis of histopathology images, employing TensorFlow as the primary machine learning framework. CNNs are specialized deep learning models that excel at image recognition tasks, as they are capable of automatically learning spatial hierarchies of features from the input images. This is achieved through the application of convolutional layers that scan the input data, preserving spatial information and detecting local patterns.

The model was designed to classify each histopathology image into one of two categories: positive or negative, based on the presence or absence of metastatic cancer. The choice of using a CNN for this task is motivated by its proven success in various image recognition problems, including medical image analysis.

The custom architecture of the model consists of four convolutional layers followed by three fully connected (dense) layers. The model employs batch normalization and max-pooling after each convolutional layer and uses dropout layers between the dense layers. The activation function used in the convolutional and dense layers is ReLU, except for the final dense layer, which uses a sigmoid activation function for binary classification.


\subsection{Exploring the Effect of Double Descent}
The number of filters and dense layers in the CNN model is systematically varied to explore the effect of double descent on the model's performance. Specifically, the \texttt{get\_model} function is modified to accept varying numbers of filters and dense layers. This led to creating multiple model architectures with different levels of complexity by adjusting the parameters such as \texttt{first\_filters}, \texttt{second\_filters}, \texttt{third\_filters}, and \texttt{fourth\_filters} for the convolutional layers, and the number of neurons in the dense layers. The most complex model capable of double descent and training on the laptop GPU was then used as an origin point to build future simplified models. This was done by removing a dense layer for each iteration of the model until there was only one dense layer left. To introduce even more simplicity into the model creating a broader vision into double descent, the same procedure was executed however the model now had one less convolutional layer.

\subsection{Optimization and Loss Function}
To ensure a fair comparison between the different model architectures, the same optimization algorithm (Adam optimizer) with a learning rate of 0.001 is used, and the same binary crossentropy loss function.

\subsection{Training Approach}
The training approach was maintained consistently across all model structures. Identical sets of callbacks were employed, such as \texttt{ModelCheckpoint}, \texttt{TensorBoard}, and \texttt{ReduceLROnPlateau}. These callbacks assisted in overseeing the model's development, preserving the optimal weights, and modifying the learning rate throughout the training process. The performance of each model was assessed using the same training and validation dataset supplied by the Patch-CAMELYON dataset.

\subsection{Performance Evaluation}
A thorough exploration of the connection between model intricacy and the double descent phenomenon was carried out by systematically documenting and comparing the performance of various models with different levels of complexity. This exhaustive analysis aimed to clarify the influence of model intricacy on the double descent phenomenon in the context of medical image evaluation using Convolutional Neural Networks (CNNs).

This research resulted in the creation of a CNN model that exhibited the double descent phenomenon from the ground up, providing valuable insights into the effects of dense layer and filter complexity on the manifestation of double descent. Through a careful examination of the model's performance, this study contributes to the comprehension of the double descent phenomenon and its practical implications in the field of medical image evaluation, ultimately laying the groundwork for more reliable and efficient automated image analysis tools in clinical settings.
