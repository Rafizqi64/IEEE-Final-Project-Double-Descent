This report analyzes the use of double descent in the analysis of histopathology images. Double descent is a phenomenon in which the performance of a machine learning model improves, worsens, and then improves again as the model's complexity increases. The analysis focuses on the use of double descent in histopathology image analysis, which is a critical task in medical diagnosis and treatment planning. The report discusses various aspects of double descent, including its causes, consequences, and implications for the accuracy and reliability of histopathology image analysis. The report also provides an overview of the state-of-the-art techniques used in histopathology image analysis and highlights the advantages and limitations of these methods. Finally, the report proposes future directions for research in this area, such as the development of new models and algorithms that can take advantage of the double descent phenomenon to improve the accuracy and efficiency of histopathology image analysis.