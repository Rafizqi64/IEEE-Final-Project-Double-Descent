This study examines the application of double descent in the assessment of histopathology imagery. Double descent is an intriguing occurrence where a machine learning model's performance initially improves, deteriorates, and then enhances again as the model's intricacy escalates. The investigation centers on employing double descent in histopathology image evaluation, an essential undertaking in medical diagnostics and treatment strategy formulation. The study delves into various facets of double descent, encompassing its origins, repercussions, and implications for the precision and dependability of histopathology image examination. Additionally, the study presents a synopsis of cutting-edge techniques utilized in histopathology image appraisal and underscores the benefits and drawbacks of these approaches. In conclusion, the study suggests prospective avenues for exploration in this domain, such as creating novel models and algorithms that capitalize on the double descent phenomenon to boost the accuracy and effectiveness of histopathology image assessment.