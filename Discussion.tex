Restricted access to GPU capabilities significantly impacted the project's results, not only extending the training duration but also limiting the ability to create intricate models. Future investigations on this subject should allocate more GPU resources, enabling a comprehensive examination of the correlation between model intricacy and the emergence of double descent.

The concept of double descent is fairly recent, with the term first appearing in 2019 (\cite{Belkin2019ReconcilingTrade-off}). This indicates considerable knowledge gaps in the field. As a result, some facets of the observed outcomes cannot be explained with absolute confidence, leaving certain aspects of the double descent occurrence as unresolved matters.

Additional elements merit further discussion and scrutiny. One aspect involves hyperparameter choices, including learning rate, batch size, and regularization methods. These factors may influence the presence and prominence of the double descent curve, and future studies should assess optimal hyperparameter configurations for various model structures in the context of double descent.

Another crucial consideration is the generalizability of the findings. This study concentrates on histopathology image analysis in relation to metastatic breast cancer. It is uncertain if the identified trends would persist in other medical image analyses or in entirely different domains. Subsequent studies should extend their scope to include diverse image analysis tasks and datasets to evaluate the resilience and generalizability of the double descent occurrence across multiple applications.

The influence of data preprocessing and augmentation techniques on the double descent curve also warrants examination. Preprocessing steps, such as normalization, resizing, and data augmentation, could potentially impact the model's performance and its vulnerability to double descent. A methodical investigation of the relationship between data preprocessing and model intricacy could offer valuable insights into the best practices for reducing the effects of double descent in real-world applications.

Additionally, it would be advantageous to explore alternative deep learning architectures, like DenseNets, Inception networks, or Transformer-based models, to ascertain if the double descent phenomenon is exclusive to convolutional neural networks or if it is a more universal feature of deep learning models. Pursuing this line of research could enhance the comprehension of the underlying mechanisms responsible for the double descent phenomenon and provide direction on the most suitable design choices for a variety of model architectures.