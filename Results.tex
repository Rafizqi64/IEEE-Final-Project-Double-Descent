\subsection{AUC Scores}
Individual accuracy and loss curves for all 6 models can be found in Appendix C. The models were prepared using the preparation code from (\cite{Veto8p361-project-imaging}) and uploaded to Kaggle to infer the test set. The scores indicate the area under the ROC curve or AUC. The private scores have been chosen due to the fact that those scores are calculated with about 80\% of the test data and therefore represent a better efficiency in the model.
\begin{table}[h]
\centering
\begin{tabular}[width=0.7\textwidth]{|c|c|c|c|c|}
\hline
\textbf{amount of dense layers} & \textbf{AUC 4 CN layers} & \textbf{AUC 3 CN layers} \\ \hline
4 & 0.8903 & 0.8954\\ \hline
3 & 0.9045 & 0.9093 \\ \hline
2 & 0.8823 & 0.9121\\ \hline
\end{tabular}
\caption{Summary of AUC Scores for each model submitted to Kaggle}
\label{AUC table}
\end{table}

\begin{figure}[!htb]
    \centering
    \begin{minipage}{0.4\textwidth}
        \includegraphics[width=\textwidth]{images/loss.png}
        \caption{Loss per model during training and validation}
        \label{Lossgraph}
    \end{minipage}
    
    \vspace{1cm} % Adjust this value if you want more or less space between the figures
    
    \begin{minipage}{0.4\textwidth}
        \includegraphics[width=\textwidth]{images/ac.png}
        \caption{Accuracy per model during training and validation}
        \label{ACgraph}
    \end{minipage}
\end{figure}

\subsection{Loss and Accuracy}
The performance during machine learning has been summarized in graph \ref{Lossgraph} and \ref{ACgraph}, and use the data found in \ref{ACtable}. These show a comparison of the models with 4 CN layers and the models with 3 CN layers in decreasing order of amount of dense layers. Model 2 (4 CN, 3 dense) which had the most prevalent double descent occurrence, had a jump in AUC score compared to models 1 and 3. Only outperformed by models 5 and 6. In the training data, there seems to be a non-linear increase in validation loss compared to train loss for each model \ref{Lossgraph} whilst there is a clear relation between training and validation accuracy that remains consistent in every model \ref{ACgraph}.