In conclusion, the study provides empirical evidence that the phenomenon of double descent can be observed in convolutional neural networks (CNNs) applied to the challenging task of identifying metastasis in breast cancer patients. The findings align with prior research on double descent, highlighting the importance of sufficiently large depths, over-parameterization, and extensive datasets to reveal this intriguing behavior.

A trend was observed between the clarity of the double descent curve and the complexity of the neural network, indicating that as the model's complexity increases, the double descent phenomenon becomes more pronounced. This observation further supports the notion that employing deeper and more over-parameterized models can lead to improved performance, despite the counter-intuitive expectation that increasing complexity would yield diminishing returns.

Furthermore, it suggests that for simpler models it suffices to determine the stopping conditions within the classical regime, as those models are often not capable of reproducing the epoch-wise double descent curve.

The research contributes to a deeper understanding of the interplay between model complexity, generalization, and performance in the context of medical image analysis. This understanding can be crucial for optimizing neural network architectures in practical applications, particularly when dealing with high-stakes tasks such as metastasis detection in cancer patients.
